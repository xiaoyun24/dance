\documentclass{exam-zh}
\usepackage{siunitx}

\examsetup{
  page/size=a4paper,
  paren/show-paren=true,
  paren/show-answer=true,
  fillin/show-answer=false,
  solution/show-solution=show-stay,
  solution/label-indentation=false
}


\everymath{\displaystyle}

\title{2024 年森铭中学第一学期期末考试}

\subject{体操二级}



\begin{document}

% \information{
%   姓名\underline{\hspace{6em}},
%   座位号\underline{\hspace{15em}}
% }
% \warning{(在此卷上答题无效)}

\secret

\maketitle

本试卷共 2 页,27 题。全卷满分 100 分。考试用时 90 分钟。


\begin{notice}
  \item 答题前,先将自己的姓名、准考证号、考场号、座位号填写在试卷和答题卡上,
    并将准考证号条形码粘贴在答题卡上的指定位置。
  \item 选择题的作答:每小题选出答案后,用 2B 铅笔把答题卡上对应题目的答案标号涂黑。
    写在试卷、草稿纸和答题卡上的非答题区域均无效。
  \item 填空题和解答题的作答:用黑色签字笔直接答在答题卡上对应的答题区域内。
    写在试卷、草稿纸和答题卡上的非答题区域均无效。
  \item 考试结束后,请将本试卷和答题卡一并上交。
\end{notice}





\section{%
  选择题:本题共 15 小题,每小题 2 分,共 30 分。
  在每小题给出的四个选项中,只有一项是符合题目要求的。
}




\begin{question}[points = 2]
    体操比赛的评分标准分为哪两部分?
    \begin{choices}
    \item 难度分和表现分
    \item 难度分和执行分
    \item 执行分和表现分
    \item 技术分和艺术分
    \end{choices}
    \end{question}
    
    \begin{question}[points = 2]
    在体操比赛中,失误扣分是根据什么标准来决定的?
    \begin{choices}
    \item 动作难度
    \item 动作完成度
    \item 裁判主观判断
    \item 动作稳定性
    \end{choices}
    \end{question}
    
    \begin{question}[points = 2]
    哪种情况会导致比赛中的动作扣分?
    \begin{choices}
    \item 动作完成流畅
    \item 动作中出现停顿
    \item 动作创新
    \item 动作协调性好
    \end{choices}
    \end{question}
    
    \begin{question}[points = 2]
    女子体操中没有使用哪种器械?
    \begin{choices}
    \item 跳马
    \item 高低杠
    \item 鞍马
    \item 平衡木
    \end{choices}
    \end{question}
    
    \begin{question}[points = 2]
    男子体操比赛中不包括哪项器械?
    \begin{choices}
    \item 吊环
    \item 单杠
    \item 自由体操
    \item 平衡木
    \end{choices}
    \end{question}
    
    \begin{question}[points = 2]
    在体操比赛中,运动员表现的哪一方面不属于评分标准?
    \begin{choices}
    \item 创造力
    \item 稳定性
    \item 难度
    \item 完成度
    \end{choices}
    \end{question}
    
    \begin{question}[points = 2]
    体操比赛中,哪种情况不会被扣分?
    \begin{choices}
    \item 动作完成后出现晃动
    \item 动作中出现停顿
    \item 动作流畅完成
    \item 动作完成过程中失去平衡
    \end{choices}
    \end{question}
    
    \begin{question}[points = 2]
    体操比赛的评分标准中,难度分是如何决定的?
    \begin{choices}
    \item 根据动作的创新性
    \item 根据动作的完成质量
    \item 根据动作的稳定性
    \item 根据动作的复杂程度
    \end{choices}
    \end{question}
    
    \begin{question}[points = 2]
    下列哪项不是女子体操比赛项目?
    \begin{choices}
    \item 高低杠
    \item 自由体操
    \item 跳马
    \item 吊环
    \end{choices}
    \end{question}
    
    \begin{question}[points = 2]
    在体操比赛中,裁判会根据什么标准扣分?
    \begin{choices}
    \item 动作的美感
    \item 动作的难度
    \item 动作的创新
    \item 动作的执行情况
    \end{choices}
    \end{question}
    
    \begin{question}[points = 2]
    哪种情况会导致比赛中的动作被视为失败?
    \begin{choices}
    \item 动作完成时间过长
    \item 动作中出现晃动
    \item 动作未完成或掉落
    \item 动作完成后停顿
    \end{choices}
    \end{question}
    
    \begin{question}[points = 2]
    在体操比赛中,动作的完成度主要是指什么?
    \begin{choices}
    \item 动作的流畅性
    \item 动作的创新性
    \item 动作的难度
    \item 动作的稳定性
    \end{choices}
    \end{question}
    
    \begin{question}[points = 2]
    男子体操比赛中,吊环项目主要考察运动员的什么能力?
    \begin{choices}
    \item 柔韧性
    \item 平衡性
    \item 力量
    \item 创造力
    \end{choices}
    \end{question}
    
    \begin{question}[points = 2]
    在体操比赛中,动作的难度分是如何确定的?
    \begin{choices}
    \item 动作的稳定性
    \item 动作的流畅性
    \item 动作的复杂性
    \item 动作的美感
    \end{choices}
    \end{question}
    
    \begin{question}[points = 2]
    女子体操中,平衡木项目的评分标准不包括以下哪项?
    \begin{choices}
    \item 动作的稳定性
    \item 动作的创新性
    \item 动作的流畅性
    \item 动作的力量
    \end{choices}
    \end{question}


\section{填空题:本题共 10 小题,每小题 3 分,共 30 分。}


    
    \begin{question}
    在体操比赛中,运动员的\fillin[完成度]和\fillin[稳定性]会直接影响最终得分。
    \end{question}
    
    \begin{question}
    动作的完成度和稳定性是体操评分中的\fillin[执行分]的关键因素。
    \end{question}
    
    \begin{question}
    在体操比赛中,失误会导致\fillin[扣分]。
    \end{question}
    
    \begin{question}
    男子体操比赛的项目包括自由体操、鞍马、吊环、跳马、双杠和\fillin[单杠]。
    \end{question}
    
    \begin{question}
    女子体操比赛的项目包括自由体操、跳马、高低杠和\fillin[平衡木]。
    \end{question}
    
    \begin{question}
    体操比赛中,动作的创新性和美感主要影响的是\fillin[表现分]。
    \end{question}
    

    \begin{question}
    裁判会根据运动员的动作流畅度、完成度和稳定性来给出\fillin[执行分]。
    \end{question}
    
    \begin{question}
    体操比赛中,动作的\fillin[复杂性]和\fillin[完成度]是决定难度分的关键因素。
    \end{question}

\section{问答题:本题共 4 小题,每小题 10 分,共 40 分。}

\begin{problem}[points = 10] 简述体操比赛中的评分标准,包括难度分和执行分的具体内容。\end{problem}

\begin{problem}[points = 10] 在体操比赛中,失误会如何影响运动员的得分?请举例说明。\end{problem}

\begin{problem}[points = 10] 解释男子体操比赛中不同器械项目的评分标准及其考察重点。\end{problem}

\begin{problem}[points = 10] 体操运动员如何通过日常训练提高动作的稳定性和完成度,以获得更高的评分?\end{problem}
\end{document}